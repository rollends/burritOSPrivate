\documentclass[twoside,a4paper]{refart}
\usepackage{makeidx}
\usepackage{ifthen}
\usepackage{hyperref}
\usepackage{tikz}
\usetikzlibrary{shapes.geometric, arrows}

\def\bs{\char'134 }
\newcommand{\ie}{i.\,e.,}
\newcommand{\eg}{e.\,g..}
\DeclareRobustCommand\cs[1]{\texttt{\char`\\#1}}
\settextfraction{0.9}
\setlength\parskip {0.15in}

\title{burritOS System Manual}
\author{Rollen D'Souza (20418141) \\
rs2dsouz@uwaterloo.ca \\
Taylor Petrick (20413951) \\
tpetrick@uwaterloo.ca \\}

\date{\textbf{CS 452 - Real Time With Bill Cowan \\ Trains 1 Demo \\ \today}}
\emergencystretch1em  %

\pagestyle{myfootings}
\markboth{burritOS System Manual}%
         {burritOS System Manual}

\makeindex
\setcounter{tocdepth}{2}
\begin{document}
\maketitle


\textbf{Noteworthy Changes}\\\\
This submission include a new document format, train tracking logic and the neccessary user space services to complete the Trains 1 milestone.

The new documentation is written in \LaTeX\ rather than a typical desktop word processor. The .tex file is included as part of the Git repository as \textit{./doc/manual.tex}, and is built into a PDF with the rest of the documentation build target. Future submissions will be patches to this document - that is, rather than resubmitting the whole document new sections relating to Milestone 2 and the Project will be submitted when required.

The \textit{./trains/} subdirectoy was added to the source code for the Trains 1 milestone. It includes a collection of files for maintaining a physical model of the train, including initial configuration data for Train 68 and Train 62.

A number of additional user tasks were also added. The system now launches a collection of tasks for each train that are used to schedule, track position and interact with that particular train. The tasks are described in more detail in \textbf{Section 6.2} of this report.

\newpage
\tableofcontents
\newpage

%%%%%%%%%%%%%%%%%%%%%%%%%%%%%%%%%%%%%%%%%%%%%%%%%%%%%%%%%%%%%%%%%%%%

\section{Compiling and Running}
A compiled version of our Trains 1 submission can be found in the following directory on the student environment:
\begin{center}
\textit{/u/cs452/tftp/ARM/tpetrick/t1.elf}\\
\textit{md5 hash: ?}
\end{center}
The source code for our submission has been shared with the CS 452 TA via the UW GitLab, and can be accessed at the following URL:
\begin{center}
\url{https://git.uwaterloo.ca/tpetrick/CS452_Project}\\
\textit{md5 hash: ?}
\end{center}
Additionally, the course code can also be downloaded as an archive from Taylor’s personal web server:
\begin{center}
\url{https://www.taylorpetrick.com/cs452/t1.tar.gz}\\
\textit{md5 hash: ?}
\end{center}
Both source code locations contain the files listed in the File Listings section of this report.

To compile the source code, simply run \textit{make} in the top level source directory. This will produce the binary file \textit{/bin/t1.elf}, which should be identical to the elf file listed above. Running \textit{make clean} will remove all build output files, and \textit{make doc} will generate documentation from the source files if Doxygen is installed on the system.

By default, code is built using a unity build technique. The makefile runs \textit{cat} on all of the source files to produce a single source file, which is then built with GCC. This can be disabled by passing \textit{UNITY=0} when running \textit{make}.

The file can be loaded and run on the TS7200 board with the following Redboot commands:

\textit{\textgreater load -b 0x50000 -h 10.15.167.5 /u/cs452/tftp/ARM/tpetrick/t1.elf}\\
\textit{\textgreater go}\\\\

\textbf{CAUTION:}  The load address has changed since our past submissions. We no longer use the default address of \textit{0x218000}, and require that code is loaded at \textit{0x50000} instead. This should be handled automatically by the linker script and \textit{.elf} file, however it may be necessary to specify the \textit{-b} parameter when loading to Redboot.

\section{Hardware Interface}
\subsection{General Notes and Usage}
The hardware layer is intended to abstract away the details of dealing with hardware perhipherals. It current wraps the timers, UARTS and interrupt controller logic in a simple interace. An implementation of the interface for the TS7200 board can be found in the \textit{hardware/ts7200} subdirectory.

The hardware layer is also responsible for providing instructions for intializing the interrupt vector table, and handling software and hardware interrupts.

\subsection{Interrupt Driven I/O}
The \textit{sysWrite} call takes a UART port and character as parameters and blocks the calling task until a transmit interrupt occurs. Since transmit interrupts continuously fire until a byte is transmitted, we disable the interrupt on the UART after transmitting a byte. \textit{sysWrite} also enables the TX interrupt so the task will be able to send data.

A \textit{sysRead} call is also provided for reading from a serial port. The call takes a UART port as its only parameter and returns a byte once an RX interrupt occurs on the desired UART. Technically sysRead could be implemented using the existing \textit{sysAwait}, however we chose to make a separate system call for readability and API consistency. Initially we also experimented with buffering input in the kernel, which required special read/write logic - the names \textit{sysRead}/\textit{sysWrite} calls are leftover from that implementation.

When using UART 1 the CTS flag is checked before transmitting data. This is done by handling the ORed interrupt for UART 1, which includes RX, TX and modem status interrupts. When a UART 1 interrupt occurs we first determine which source (RX, TX, Modem) it is by reading the UART’s IRQ status register. If a modem status interrupt is triggered, the interrupt handler queries the UART’s CTS register and updates the kernel’s CTS flag. If the interrupt is an RX or TX interrupt, the event table is checked to see if any tasks are waiting to read or write from UART 1. TX interrupts are only processed for UART 1 if the CTS flag is set to 1 UART 2 does not require a CTS flag, so the ORed interrupt trigger isn’t used. The event handler processes the raw TX and RX interrupts for that UART in the same way that the timer events are handled.

\section{Kernel Design}

\subsection{Memory Layout}
Task stacks are placed at address \textit{0x01000000} and higher. Task stacks are created in 16kb blocks. The stacks are managed by a \textit{MemoryAllocator} that uses FIFO queues to give out and reclaim stack blocks. The first word on the stack is its block id; the remaining space is used by the task as desired. Storing the id to the stack serves a dual purpose - it makes reclaiming the stack easier, and also act as an indicator for stack overflows. A block id that seems corrupted when the stack is reclaimed by the allocator is a strong indication that a neighboring stack overflowed.

The system has more memory blocks that tasks, so the additional blocks can be access by tasks using the \textit{sysAlloc()} and \textit{sysFree()} calls. These methods also return \textit{16kb} blocks, and serve as a basic memory page allocator.

\subsection{Data Structures}
All data structures in our implementation fall into two categories: flat arrays and circular queues. The circular queues are implemented using a memory block, a length variable and head/tail indices indicating the start and end of the queue. The length and memory pointer are passed in when the queue is initialized. This allows the size of the queue to vary based on the context (e.g. small send queues but large ready queues). Our kernel requires that queues have a length that is equal to a power of two. This is to avoid using modulus function from libgcc. Instead, since the queues are a power of two in length, a bitwise \textit{and} can be used to ensure the queue indices wrap around to 0. For example:

\begin{verbatim}
    head = (head + 1) & (length - 1)
\end{verbatim}

The priority queue used for scheduling of ready tasks is a special data structure that contains a set of circular buffer queues. One queue is used for each of the priorities. Kernel 1 and 2 only used three priorities, however in Kernel 3 the number of priorities was increased to 32. A 32 bit integer is used to determine which queues have tasks in them. Bit twiddling is done to compute the highest priority queue with an element in it; the code to do this was taken from \url{https://graphics.stanford.edu/~seander/bithacks.html}. It emulates a count trailing zeroes (CTZ) instruction since ARM v4 lacks CTZ/CLZ.

TaskDescriptors (or otherwise known as TDs) contain the information the kernel needs to schedule tasks and perform system calls. Non essentially information such as the PC and SPSR are stored on the task stack. The kernel stores a TaskTable which has a fixed size array of task descriptors and queue of free descriptors. At the start of the kernel, all descriptors are in the free queue. When a task is created, it pops from the queue and uses the available task descriptor. When a task exits it returns the descriptor to the queue so it can be reused later on. A generation id is stored along with each task id, and gets incremented each time a task is reclaimed by the TaskTable. The TD recycling feature is not used for Kernel 1, but it is included in the code since we plan to build on it in future assignments

TDs themselves are designed to be tightly packed. Task priority is stored as a single byte rather than a full 32 bit value, for example. TaskIDs are 16 bit values, where the first byte is the generation number and second byte is the actual task id and index into the TaskDescriptor array. The task ids are one-based, since task 0 will eventually be used to identify the null process. At the moment, our kernel is defined to support a maximum of 64 tasks. It can support as many as 255 tasks and 255 generations, however it seems unlikely that we will come close to either of these values.

The last important data structure is the kernel data struct. Kernel data is stored as a global that is only used in  kernel functions (bootstrap, schedule and system call). It contains the task table, a pointer to the active task descriptor and the priority queue and queue data.

\subsection{Control Flow}
The bulk of the kernel logic is controlled in the assembly file \textit{kernel/start.s}. The kernel begins by initializes the interrupt vector table. It then calls the \textit{bootstrap()} function, which initializes all of the kernel data structures and creates the first user task. The user task is entered and runs until an SWI instruction is executed.

When a system call is made the program branches to the \textit{\_systemCall} label in \textit{kernel/start.s}. Since the \textit{SWI} call is wrapped in a C function, GCC automatically places the parameters into \textit{R0-R3} and onto the user stack. To minimize the amount of memory overhead, the system call is processed almost immediately after entering the SWI handler. This is done by calling \textit{systemCallHandler()}, which is performed in System Mode so the user stack is available. The return value from the function is stored in \textit{R0} by GCC. After the call completes the user registers are stacked, including the new \textit{R0} value. When the user task registers are unstacked the next time the task runs, the \textit{R0} value will be restored so the task can inspect the system call return value.

After stacking the registers the kernel calls \textit{schedule()} which updates the task descriptor of the current task, re-enqueues it if necessary and selects a new task from the priority queues. The new task’s stack pointer is returned and the task is entered, thus repeating the process. If the new stack pointer is 0 the kernel will branch to the \textit{kernelEnd} handler to perform cleanup and exit the program.

When an interrupt is fired the control flow is very similar. The interrupt causes the code to jump to the \textit{\_interruptCall} label, which is very similar to \textit{\_systemCall} discussed above. The primary difference is that the interrupt handler calls the  \textit{interruptHandler()} C function instead of processing a system call. The other functionality, such as stacking user registers and scheduling a new task, is the same.

It is worth noting that the task registers are stored in the following layout on the task stack:

\begin{verbatim}
    stack+0x0       SPSR
                    Task LR (R14)
                    R0
                    R4
                    ..
                    R12
    stack+0xA       Task PC (R15)
\end{verbatim}
This somewhat out-of-order stacking order was chosen to make the context switcher more compact.

\subsection{Event Handling}
The \textit{sysAwaitEvent(event)} primitive allows tasks to wait for interrupts. The supported events are defined in \textit{kernel/event.h}. The kernel data struct has a table that maps event types to the task ids waiting on the events. Only one task can wait on an event at a time, which we think will be sufficient for the remaining coursework.

When an interrupt occurs the system jumps to the label \textit{\_interruptCall}. This works much like the swi handler, however instead of calling the syscall function it calls \textit{interruptHandler()} to process the interrupt. The interrupt handler checks the event table to see if a task is waiting on the event. If a valid task id is found, the task is unblocked and the table entry cleared, otherwise the interrupt is consumed without taking any additional action. Like the swi handler, an interrupt results in a call to the scheduler after interrupt processing is complete.

\subsection{Message Passing}
Messages can be set between tasks using the \textit{sysSend} and \textit{sysReceive} system calls, both of which are blocking. After receiving a message a task must reply to its sender using the \textit{sysReply} primitive.

By default, no deep memory copies are done in the kernel. All messages are at most 1 word in length, however that word can be a pointer to a large buffer of memory if desired. When the kernel sends a message it copies the message type and the 1 word message body to the receiver’s message envelope. It is up to the receiving task to make a copy of the memory using the \_memcpy function. This design makes the kernel extremely lightweight, and also eliminates the need to do a memory copy if the message data is used in a read-only way. For safety reasons, this design is unsuitable for general purpose operating systems, but in a control RTOS it offers performance benefits.

For the performance benchmarks with 64 bytes a full memcpy is done upon receiving a message and after receiving a reply. The memcpy implementation was written from scratch and uses load/store multiple instructions to improve copying speed.

\subsection{Performance}
Performance timing is reset with a call to \textit{sysPerfReset()} - this sets all task performance counters to 0 and resets the timer used by the kernel to track performance. The kernel uses the 40-bit debug timer for this process since it’s clocked at around 983 Khz. This offers slightly better timer resolution that the standard 16 and 32 bit timers, which can only run at a max of 508 Khz.

Calling \textit{sysPerfQueryP(tid, mode)} will return the percentage of execution time spent on the task tid. The mode parameter can be \textit{ePerfKernel}, \textit{ePerfTask} or \textit{ePerfBoth}. Using \textit{ePerfKernel} returns the time spent by the task in the kernel, and \textit{ePerfTask} returns the time spent in task code. The \textit{ePerfBoth} option is useful be cause it adds the kernel and task performance together before computing the percentage - this yields slightly higher precision results in most cases since the numerator of the computation is larger.

The \textit{sysPerfQueryT(tid, mode)} system call queries the run time microseconds for each of the cases listed above, rather than as a relative percentage. Additionally the mode value ePerfTotal can be passed in to get the total run time in microseconds since the last call to \textit{sysPerfReset()}.

The performance timings are implemented by storing two unsigned integers at the end of each task descriptor. When a system call or interrupt occurs, the active task has its task mode counter updated by the current timer delta. Then, right before the scheduler returns, the last active task has its kernel mode counter updated by the timer delta. Both updates reset the timer delta. This means that task mode timings are the amount of time spent from the return statement of the scheduler until the entry point of a system call or interrupt, and kernel mode timings are the amount of time spent inside the syscall/interrupt and scheduler. This is not a perfect representation of task/kernel time, but is close enough to be useful.

The following output was produced using the data recored by the performance primitives:
\begin{verbatim}
    TaskId  Tsk %   Krnl %  Ttl %   Tsk us      Krnl us     Ttl us
    0x00    46.17   50.68   96.86   1,156,700   1,269,741   2,426,442
    0x01    00.26   00.00   00.26       6,623          64       6,687
    0x02    00.00   00.00   00.00          28          43          72
    0x03    00.03   00.05   00.10       1,053       1,509       2,563
    0x04    00.03   00.05   00.10       1,075       1,539       2,614
    0x05    01.21   00.00   01.22      30,915          91      31,012
    0x06    01.22   00.00   01.22      31,209         114      31,324
    0x07    00.55   00.00   00.55      14,049          57      14,107
    0x08    00.36   00.00   00.36       9,366          42       9,408
    0x09    00.17   00.00   00.17       4,691          26       4,718

    Total run time: 2,552,806us ~ 2.552s
\end{verbatim}

\section{System Calls}
\subsection{System State}

The \textit{System State} calls are used to monitor the state of the kernel. Currently there are two calls in this category.

\textbf{sysRunning(void)} - Returns 1 if the system running flag is set, else returns 0.\\\\
\textbf{sysShutdown(void)} - Sends a shutdown signal to the kernel, and sets the running state to 0. Does not force the system to terminate; processes must check the running flag with \textit{sysRunning} to determine if they should continue executing.

\subsection{Task Management}

\textit{Task Management} system calls are used to create, destroy and query tasks. There are a total of 7 task related system calls.

\textbf{sysExit(void)} - Exits a task and returns the descriptor and stack back to the kernel. This system call is deprecated; tasks will perform this operation on their own once the task function returns.\\\\
\textbf{sysPass(void)} - Yields control of the processor to the kernel. Effective performs a no-op, however since the scheduler is invoked a different task may be entered.\\\\
\textbf{U16 = sysCreate(priority, entry\_point)} - Creates a new task with a given priority and code entry point. The new task's id is returned.\\\\
\textbf{char* = sysName(tid)} - Returns a pointer to the string name of a given task.\\\\
\textbf{U16 = sysTid(void)} - Returns the id of the current task.\\\\
\textbf{U16 = sysPid(void)} - Returns the id of the current task's parent.\\\\
\textbf{U8 = sysPriority(void)} - Returns the priority of the current task.

\subsection{Messaging}

\textit{Messaging} system calls are used to perform inter-task communication. Currently only blocking ITC is supported.

\textbf{sysSend(tid, *msg, *reply)} - Sends a message to task, and provides a memory location in which the reply should be stored. Blocks the caller until a \textit{sysReply} is sent by the receiver.\\\\
\textbf{sysReceive(*tid, *msg)} - Receives a message and stores the sender id and message body. Blocks the caller until a message arrives.\\\\
\textbf{sysReply(tid, *msg)} - Sends a reply message back in response to a \textit{sysReceive}.

\subsection{Performance}

The \textit{Performance} calls are used to monitor system performance. See \textbf{Section 4.6} for details on performance monitoring.

\textbf{sysPerfReset(void)} - Resets performance tracking on all tasks.\\\\
\textbf{sysPerfQueryP(tid, mode)} - Queries performance of task for a given mode, as a percentage of total runtime.\\\\
\textbf{sysPerfQueryT(tid, mode)} - Queries performance of task for a given mode, as an absolute time in microseconds.

\subsection{Event Handling and I/O}

\textit{Event} system calls are used to block on hardware interrupts and interact with hardware peripherals. See the \textbf{Section 2.2} for details on interrupt driven input and output. All Event calls cause tasks to be blocked on hardware interrupts.

\textbf{sysAwait(event)} - Blocks the calling task on a specific event.\\\\
\textbf{char = sysRead(port)} - Blocking read from a UART port.\\\\
\textbf{sysWrite(port, byte)} - Blocking write to a UART port.

\subsection{Memory}

The burritOS system does not use or provide heap allocation, however it does have a generic memory block allocator. The allocator is used to give out stack blocks; extra blocks are available via memory system calls. No protection or ownership verification is done on the blocks.

\textbf{U32* = sysAlloc(void)} - Returns a pointer to a 16kB - 1 word memory block. The first address holds the block id, so the memory returned begins after that location.\\\\
\textbf{sysFree(block*)} - Frees a memory block pointer for reuse by the system.

\section{Debugging and Assertions}
 Almost all hardware and kernel functions return a status code and perform safety checks including out-of-bounds access, invalid memory addresses and invalid parameter values. Assertions are used liberally to validate both kernel and user-space code. When an assertion is throw, a debug message is printed:

\begin{verbatim}
    ....................
    ....................
    ......______ .......
    ...../      \ \.....
    ..../  x  x  \n\....
    ...|   ____   hn|...
    ...|  /    \  hn|...
    ....\_________\/....
    ....................
    .....SAD..TACO......
    ....................

    Assertion in `kernel/scheduler.c` on line 18
    Current Active Task is 0x06
\end{verbatim}

Assertions, return codes and bounds checking can be enabled/disabled using a collection of preprocessor flags. When return codes are disabled functions return void, otherwise they return a signed integer. These flags are:\\
\begin{verbatim}
    NULL_CHECK              Include null pointer checks on memory address
    BOUNDS_CHECK            Include bounds checking
    RANGE_CHECK             Include memory address range checks
    DEVICE_CHECK            Include device register checks (e.g. UART, timers, etc)
    ASSERT_BUILD            Enabled assertions
    RETURN_CODES            Enable return codes
    KERNEL_PERF             Enable kernel performance primitives
    KERNEL_PERF_VERBOSE     Enable verbose performance updates from the perf Task
\end{verbatim}

These flags enable a set of error and assertion macros that are normally disabled. For example, for null checking:

\begin{verbatim}
    #ifdef NULL_CHECK
        #define IS_NOT_NULL(x) do {if ((x) == 0) return ERROR_NULL;} while(0);
    #else
        #define IS_NOT_NULL(x)
    #endif
\end{verbatim}

It is up to the kernel programmer to use the macros in their code, so we have not yet achieved full coverage. This is an area that will be improved in future iterations of the kernel.
\section{Userspace}
\subsection{Services}
The general purpose user-space tasks provide important services on top of the kernel. All of these tasks provide a named interface that can be discovered using the \textit{Name Server}. The \textit{Name Server} is assumed to be have a task ID of 2.

\subsubsection{Name Service}

\textbf{nsRegister(taskName)} - Registers the caller task with the name provided.\\\\
\textbf{nsWhoIs(taskName)} - Looks up the \verb~TaskID~ of the task registered with the provided name. Will block until \textit{all} task names have been registered.\\\\

The \textit{Name Server} task triages and responds to requests to find the task ID of a "named" task. Two service methods are provided as the interface to the name service. This service provides the ability for tasks to find other operating system services. The current implementation requires that all possible registrations be made before any lookup (WhoIs) requests are served. As a consequence of our kernel development tenets, we chose enumerated values to represent task names instead of strings. This reduces complexity in the user space code of the nameserver. An enumerated list has a well-defined maximum meaning the nameserver can maintain a constant-sized array to store registered IDs.

\subsubsection{Clock Service}

\textbf{clockDelayBy(clockServerID, ticks)} - Delays a task by \verb~ticks~ times 10 milliseconds. The delay is guaranteed to an interval of plus or minus 10 milliseconds.\\\\
\textbf{clockLongDelayBy(clockServerID, longTicks)} - Delays a task by \verb~longTicks~ times 150 milliseconds. The delay is guaranteed to an interval of plus or minus 150 milliseconds.\\\\
\textbf{clockTime(clockServerID)} - Returns the current time reported by the \textit{Clock Server}, in multiples of 10 milliseconds (\verb~ticks~).\\\\
\textbf{clockDelayUntil(clockServerID, time)} - Delays a task until the \textit{Clock Server} reports that [10 millisecond] tick \verb~time~ has passed. This method is implemented using calls to \verb~clockTime()~ and \verb~clockDelayBy~ and is therefore more susceptible to error.\\\\

The clock server is first instantiated by the initial task at priority 1 (one level short of the highest possible priority). It has the responsibility of: creating the clock notifier (at priority 0), handling delay requests, unblocking tasks whose delay time has passed. The notifier is at a priority higher than the server to ensure that the ‘tick’ interrupt is handled without blocking any other driver interaction; this model is useful in future parts of the project where various interrupts will compete for operation.

The clock server must be as responsive as possible in order to service the timer reliably for user tasks. \textit{BurritOS} implements a sorted linked-list data structure to facilitate fast dequeuing of delayed tasks. A task that wishes to delay a certain number of ticks is enqueued in the list in sorted order. Additionally, each node in the list retains a count of how many ticks to delay relative to its parent. This structure has the property of requiring the clock server to only decrement the head of the list during an uneventful timer tick. It is trivial to observe that a ‘dequeue’ tick occurs when the linked list has a set of nodes at the head with 0. The worst-case cost occurs during insertion with $O(n)$. Retrieval is at worst $O(k)$ where $k$ is the number of tasks that dequeued in one tick. During an uneventful tick (no dequeue operation), the cost is constant (decrement and check the head).

The primitives the clock server exposes to user tasks include: \verb~clockDelayBy~, \verb~clockTime~, \verb~clockDelayUntil~. \verb~clockDelayUntil~ is implemented as a function of the other two primitives for the purpose of this assignment. The other primitives wrap messages to the clock server. \verb~clockTime~ is replied to as soon as possible. \verb~clockDelayBy~ is replied to when the delay time has passed, unblocking the task in real-time.

The notifier is a simple ‘await-send’ loop. A more complicated architecture - courier and warehouse - was not used for this assignment since our message passing mechanism is fast and therefore we expect a simple architecture to provide reasonable response time. The notifier does have the additional task of killing the \textit{Clock Server} cleanly during system shutdown.

\subsubsection{Hardware I/O Servers: Terminal}

\textbf{printf(formatString, ...)} - A partial implementation of the C standardized \verb~printf~. It is does not wait until the output has been flushed.\\\\
\textbf{getc()} - Blocks and consumes a character from the \textit{Terminal Server}. Only one task should use this as the input is not duplicated for each task.\\\\

We chose to implement a slightly different interface for input/output than the kernel specification discusses. We deemed it costly to use \verb~putc~ directly since we generate long strings of text at specific locations using cursor commands. To accomodate this requirement, we implement \verb~putStr~ which sends whole strings to the \textit{Terminal Server}. This is used by userspace \verb~printf~ implementation in order to reduce the number of system calls and messaging overhead.

The terminal server architecture is as simple as possible. The server creates two notifiers: one for input and one for output. The server also maintains buffers for both input and output. Originally we designed a more elaborate structure (warehouse-courier model) to handle high bandwidth but observed that this structure was unnecessary given our messaging round-trip time.

\subsubsection{Hardware I/O Servers: Train}

\textbf{trainStop(trainServer)} - Commands the board to send an emergency stop to the track.\\\\
\textbf{trainGo(trainServer)} - Sends a Go to the track.\\\\
\textbf{trainSetSpeed(trainServer, train, speed)} - Sets a train to a given speed.\\\\

The input and output components of the train are implemented as one I/O server. Complexity was introduced as needed for the switches and sensors. The switches need delays of 160ms between every command. In order to implement this, we took advantage of the courier design. An independent switch server handles switch requests. This switch server creates a courier that communicates to the actual train I/O server. The trick is to implement a delay inside the courier, hiding the delay from both servers and, more importantly, from any other request that may come to the train server (set speed, reverse). This model also allows the switch server to easily determine when to send a turn-off-solenoid request; this event fires when the courier arrives at the switch server without any pending switch commands.

Complexity was introduced in sensor reading for performance reasons. Since the train UART is slow, we would like to send requests for a group of sensors. Unfortunately, that means tasks would send requests for more data than they’d need. Instead we choose to have a sensor server which buffers all the sensor data as fast as it can while passing along specific sensor information when clients request it.

Tasks will almost never directly talk to the \textit{Train Server} and instead will go through another more job specific server for the command they are sending. For example, a \textit{Train Switch Server} for throwing a switch.

\subsection{Tasks - Milestone 1}
Milestone 1 provides the ability to track the position of a single train on the track. This includes ensuring the train can stop at an arbitrary point on the track. We chose to stop the train only at sensors using a methodology that easily generalizes to arbitrary points on the track graph.

\subsubsection{Switch Office}
The \textit{Switch Office} (server) maintains a calendar which, in a sense, represents temporal ownership of a given switch. The server handles allocations to throw a switch into a given state at a given time and rejects invalid allocations --- i.e. a switch that was thrown into a state by someone else in that period. In order to facilitate requests effectively, the server spawns workers that utilize the \textit{Clock Service} to throw a switch at the allocated block of time.

This feature is not important for this milestone but will be necessary in future submissions.

\subsubsection{Train Commander}
\label{sec:traincommander}
The train commander fields commands from the command line input to train tasks. Every train dispatches a \textit{Locomotive Radio} courier to register a particular train ID with the \textit{Train Commander}. This model permits a train to block input to itself but not to other trains.

\subsubsection{Locomotive Task}
Every train is controlled by a single task, known as the \textit{Locomotive}, whose purpose is to create and facilitate communication between useful jobs. For this milestone, this consists of: commanding the train (through command input), positioning information, and prediction.

The \textit{Locomotive} creates the following subsystems:
\begin{itemize}
    \item{ \textit{Locomotive Radio}: Courier that transfers commands from central \textit{Train Commander}. These are commands entered in the command prompt.}
    \item{ \textit{Locomotive GPS}: A server that maintains and updates position information on the track graph using sensor reads. Two couriers facilitate communication between the \textit{Locomotive} and \textit{Locomotive GPS}. }
    \item{ \textit{Physics Tick}: A notifier that provides a reasonably reliable physics integration timestep. Used by the \textit{Locomotive} to calculate stopping distance through numerical integration of the (current) velocity. }
\end{itemize}

The reasoning behind the decision to implement the \textit{Locomotive GPS} as a server with courier-communication is non-trivial. The driving motivation lay in the guarantee that the \textit{Locomotive} never wastes time. Missing a physics tick (or delaying a physics tick) can result in inaccurate calculations. Delaying an urgent user command due to a sensor read can lead to catastrophic results. It only follows that the \textit{Locomotive} task should almost never block. This is, of course, impossible without a very extensive design since it has to send commands to the driver (which blocks temporarily). Nevertheless, we attempt to minimize time spent blocked using couriers to transfer commands and position/prediction updates.

\subsection{Tasks - Milestone 2}
Milestone 2 extends from the previous milestone by adding additional functionality to the Train processes. The major (though simple) change involved adding track ownership. All other modifications extend from track ownership by simply considering which train owns what track piece. Our policy to prevent collisions simply requires that we can calculate the stopping distance of any train and allocate track ahead. Once this requirement is met, trains (the \textit{Locomotive} task) must only work with track nodes (branches, sensors) that they own. As long as all trains follow these policies, they cannot possibly collide. It is easy to show that any operation that violates these policies will likely lead to a system failure.

\subsubsection{Track Ownership}

\textbf{trainAllocateTrack(train, requestList)} - Revokes any previous allocation and attempts to allocate a new set of nodes to the train. Assumes the list be a contiguous sequence of nodes in the graph. \\\\
\textbf{trainWhoOwnsTrack(nodeId)} - Asks the \textit{Track Manager} which train owns the track node specified. \\\\

Track ownership is stored by an independent server known as the \textit{Track Manager}. The \textit{Track Manager} does not provide any protection to prevent ownership issues (such as leap-frogging) but instead provides a framework that supports ownership policies. That is, it serves as a central location to perform and store ownership information. The server provides an interface that permits a \textit{Locomotive} to allocate a sequence of nodes (path) atomically. Any allocation is preceded by a full revocation of all nodes owned by the requestor train. This gives the \textit{Locomotive} task freedom to only consider nodes it needs instead of those that it may have used in the past. This operation is atomic with respect to the \textit{Track Manager} ensuring that all contiguous path requests will remain contiguous.

\subsubsection{Locomotive Task - Path Finding}
Path finding is not treated as a separate task and is instead treated as an operation in the \textit{Locomotive}. The Dijkstra Algorithm is used to find the shortest path from any given node to any other node. It is performed at every sensor hit easily handling failed branches or a missed target sensor. Reverse path finding is not performed yet.

Path finding doesn't modify the branch state of the board since this can cause conflicts with other trains. Instead, as part of the ownership process of the train, the \textit{Locomotive} switches branches that it owns to the correct configuration. It uses physics calculations to determine whether the branch can be switched in time. If it cannot, path finding will be performed again at the next sensor and the \textit{Locomotive} will attempt a new approach towards the target sensor.

\subsubsection{Locomotive Task - Sensor Attribution}
Sensor attribution extends from the previous milestone by, once again, simply looking at the node ownership. Since the \textit{Locomotive} task owns nodes (instead of just branches, or edges), the \textit{Locomotive} can determine if the train actually should consider itself at a given sensor. Under the assumption that trains only travel on track they own, a fired sensor must be caused by the train that owns that sensor. Any other situation describes a context in which tracking of trains has failed.

\subsubsection{Locomotive Task - Collision Avoidance}
Collision avoidance simply uses track ownership, as described in the \textit{Track Manager}. The \textit{Locomotive} always attempts to allocate enough track ahead of its current position that permits it to stop with some extra room. If there is a conflict (i.e. a failure to allocate track), the \textit{Locomotive} task stops the train and then delays for a random amount.

The random amount is placed in order to give a chance for one train to own track in the reverse direction and reduce the chance of a long-lasting live-lock. If the train can reallocate after this time has passed, it then resumes in the same direction. Otherwise, the train will reverse and try again.

Unfortunately, there are occasions in which the train loses tracking during a reversal. This bug shall be fixed in the final submission.

\section{Train Modeling}

\subsection{Static Model}
A variety of data collection experiments were performed when developing the physical model for the trains. This included starting distance, stopping distance, inter-speed acceleration and steady state velocity. Only the steady-state velocity and stopping distance were used for the first demo, since the train didn't need to change speeds. Error in tracking from the lack of starting acceleration model was ignored during the demo.

For the steady state velocity, a one-to-one mapping of train speed to velocity was produced. For example, the following map was used for Train 62:
\begin{verbatim}
    Speed  5   ->   224376 um/mt
    Speed  6   ->   275931 um/mt
    Speed  7   ->   335343 um/mt
    Speed  8   ->   389936 um/mt
    Speed  9   ->   441837 um/mt
    Speed 10   ->   491406 um/mt
    Speed 11   ->   541752 um/mt
    Speed 12   ->   589662 um/mt
    Speed 13   ->   632801 um/mt
\end{verbatim}

The units on the velocity are micrometers per megatick. Since we used the 983 Khz timer it was easiest to perform the calculations in ticks rather than in seconds. The computations were done using fixed point integers, hence the use of micrometers instead of millimeters.

Note that speeds 1 - 4, and 14 were ignored. We found that speeds below 5 were too slow to reliably tranverse the track without getting stuck at dead zones. Speed 14 produced strange behavior for some of the trains. For example, for Train 62 it had the same velocity as speed 13, but a different acceleration profile.

Modeling acceleration was done by mapping all valid speeds, including 0, to all other possible speeds. This required a 10x10 mapping array. The acceleration functionality was not used during the Trains 1 demo, however an example acceleration table for Train 62 is listed below:

\begin{verbatim}
    Speed 5 ->  0       -113256 um/mt^2
    Speed 5 ->  5             0 um/mt^2
    Speed 5 ->  6         92468 um/mt^2
    Speed 5 ->  7        177639 um/mt^2
    Speed 5 ->  8        192717 um/mt^2
    Speed 5 ->  9        205884 um/mt^2
    Speed 5 -> 10        213830 um/mt^2
    Speed 5 -> 11        237231 um/mt^2
    Speed 5 -> 12        255761 um/mt^2
    Speed 5 -> 13        252275 um/mt^2
\end{verbatim}

The units on the accelerations are micrometers per megatick squared.

The actual stopping distance calcuation in the Trains 1 Demo code is hacked together, since we had to switch trains at the last minute. As such, the \textit{trainPhysicsStopDist} method directly computes the stopping distance rather than going through the proper table lookups. This computation also takes into account overshoot by adding a factor of $v*60ms$.

\subsection{Dynamic Model}

Run time corrections are made to the model to improve accuracy. During the first ten sensor triggers the velocity model is updated to account for any discrepencies between the train and the static model. On subsequent sensor triggers, any timing error was stored on the sensor node so that future predictions could account for it. The assumption was that in most cases, if the train was late to a sensor once it would have a decent chance of being late again.

The combination of the static model and run time corrections to the dynamic model was sufficiently accurate for the demo. We were generally able to stop the train within a few centimeters of the target location and accurate predict the arrival time at sensors with less than 30ms of error.

\section{File Listings}
The following listing contains the name and hash of all of the source files in the current project:
\begin{verbatim}
    7a570500eaaa3aaece0e2c3ab932806c  common/queue.h
    5111780c98f90c807e28c9051a957bd1  common/priorityQueue.h
    236769e5da7bac28b735653c61180294  common/random.h
    6e4a6dc8741dbc70e7f58f6cb97de87f  common/common.h
    f5f8b0aba1d08886d21a252ed70af1fb  common/random.c
    dbc1e1f655ec2eb5634d253a4e225901  common/error.h
    fa27cfc4310ef2f2a4aa21a263f0ac15  common/utils.h
    4ad96714a59e326dd08f49fa87ac8edf  common/memcpy.s
    1aca0ab83913d5a7432db2ce8b3e46b4  common/timerState.h
    0a75e9ea015602357e05992d269d1cae  common/memcpy.h
    9a51fd55cbaa0561763a09336f2a2eea  common/queue.c
    a7d11483d9452d5d8f9f784bb26e8209  common/types.h
    65491591f39cd860d49c262b3541a11e  common/memset.c
    7fdf99d4a9ea148c4e8d4d8b2777404d  common/vaList.h
    51f48f7fffd96b988d0d424119850596  common/string.c
    0eb10e8842dc414bcd499341a23cc5db  common/string.h
    06d14c9b85a344d1794fd1de3aca1cc1  common/memset.h
    adcf35544f06b19164f55b953995d1ff  common/priorityQueue.c

    d5f38025b527f774835847d5f79ad85b  hardware/uart.h
    6c5930b3f894fe526814ea0ce188ed4e  hardware/timer.h
    b36394bd275b36728ac6122e45a7f142  hardware/ts7200/start.s
    6055b18f1d4b2f2d6a200ff8cbcbeebf  hardware/ts7200/ts7200.h
    87306d757b7eda3d725cad19dc6bf919  hardware/ts7200/interrupt.c
    4e90444ef3fbdfb1078e8682dee3084e  hardware/ts7200/uart.c
    20421e2c6a4e0d22bce2d9c812680b34  hardware/ts7200/timer.c
    ac2247029448237de4c4590f27698a85  hardware/ts7200/vector.s
    c71fbd7b3723c4bf4d0d104902b01487  hardware/memory.h
    4541004aefa37b3c3e19d8c184deb2f6  hardware/hardware.h
    98f10a0ba0008da70ef80f36cd2a9031  hardware/interrupt.h

    3d29dd33b7f3526ae350378f872644b2  kernel/taskTable.h
    59e283c66c5a66e627e33f3ef39a6f58  kernel/event.c
    26f5ecdbe74ff55fddc97b6127cee60e  kernel/event.h
    f8a30fe67fb0ab3ae7905cf429aaf5a0  kernel/kernelUtils.h
    b9ccd6cc031e6d339d7bc7bd18f7f895  kernel/assert.c
    00ada8b6e2ff6aa0c1c79396609c8927  kernel/systemCall.c
    7617ed11988f83aa8440410517efd7f7  kernel/print.c
    46216ee62cc2a38d1194ca9c435fe0af  kernel/systemCall.h
    027b87b518651720441978c985804ce4  kernel/kernel.h
    4b77d920ca414341407ba26669285f02  kernel/memoryAllocator.h
    c1b8f89e01d797c5b606726807ac6e4b  kernel/kernelUtils.s
    cddeb3d30d579f5cddb6e05a24ff96dd  kernel/taskTable.c
    74845107469cf4a11eb30e63ccedb531  kernel/bootstrap.c
    be97388a9577fb156157cd5e3f7df771  kernel/scheduler.c
    caac750f901b606a3b2f951f3483d091  kernel/assert.h
    e1f4e45f2752c7d539018a020a6ba8e6  kernel/taskDescriptor.h
    f6c4b6303c324dcfecf1e6333b7ba02c  kernel/kernelData.h
    c7abe47341dd218cace01e289281a9fe  kernel/memoryAllocator.c
    60aed85beecdff6275fe0e71a48c6df2  kernel/config.h
    269831580f0cb96b0128c175c677d590  kernel/message.h
    6a21b1817843435936a212ce8f594f85  kernel/print.h
    8169ff0345038b3337e244491a014d07  kernel/kernelData.c

    1352f3743944badbb8c2399e6fb2ccd4  trackdata/new/track_data_new.h
    cc1cbe679f12e26e95e6580ca063ebe5  trackdata/new/track_data_new.c
    bff8429e47061fcbd57199fd54cfd0b2  trackdata/trackData.c
    b5bb94c81caa1e24b4387d3f293859d0  trackdata/trackData.h
    ded037f2ebd6fca0c9ab6f10852f4d35  trackdata/trackNode.h
    f4afacf8d77358966835c9538d7ff2bb  trackdata/legacy/model.c

    04750c307ce96072bc6893933fdb7b80  trains/trainInit.h
    7b356af8288465da34caaf7eae16a035  trains/trainPhysics.c
    5b3c0f265f184d7e95a1b8bedff3c1b4  trains/train62.h
    5483d30b360258dd3d1da6a34a3ca7ac  trains/trainPhysics.h
    1161c63d1c0f24c6f1b4c173d39d750d  trains/trainMath.h
    8d75585b5fab2f967ff68ee6ee41a388  trains/train68.h

    ba7ee9f4d08d785fd60c7e5afadfb1f2  user/PerformanceTask.c
    1dc31a1c273532052c366b1b8fa8c586  user/trains/TrainCommander.c
    f9674818e242137b06d4b62c0cdbfd1a  user/trains/TrainYard.c
    bb372bd6b7586f68f30ea287b753e5a8  user/trains/TrainCommander.h
    19186aaece2ba13ed4f1d6785933bc30  user/trains/trains.c
    bc41d8ef222498b059f4c89d78e728de  user/trains/trackData.c
    3d4b10ab39c266385db62beca0108866  user/trains/trackData.h
    a51fee6be16c32e8298b2e8d6739cd3f  user/trains/trackNode.h
    c4eda1eb8b6fd29989447793dc730dff  user/trains/SwitchOffice.c
    8445ae5c02c17771caaf23e655b600d7  user/trains/SwitchOffice.h
    b384e3973c8782620b966c35b77308b9  user/trains/trains.h
    cd12a73f7ce5746e18d7a5e375ab506b  user/trains/TrainYard.h
    d74b948e07b52e5de119f7e3367df9b0  user/SensorDisplay.c
    63bb9b7bc6d198c30c328b2f2e0850f8  user/InitialTask.h
    a833fcdae45fb79e86cbfe8aa52b4553  user/Locomotive.h
    87b141bec054ee126714e0ddc6d5c3ef  user/DigitalClock.c
    a172fa9d55a825ef2dfa891b91279ec5  user/servers/TrainDriver.c
    6fd529e84ccf8f4264e0c9d22870e626  user/servers/TerminalDriver.h
    304b576694eb39739e46f5293f6d6cec  user/servers/TrainSwitchServer.c
    5ce850a0ca90fbba0277955dfc8d74db  user/servers/NameServer.c
    fda7de6c69066a709f71851d409038c8  user/servers/ClockNotifier.h
    b1228054cb0be753d947d59480718fa5  user/servers/TrainDriver.h
    7c8d6a256a3bd5ee748c79806a3e2225  user/servers/NameServer.h
    24e0865f484341092ad222ebfb47f8ea  user/servers/TerminalDriver.c
    557bd2106756c181127d7a928b496979  user/servers/TrainSensorServer.c
    368242d16e123c6910bce9bbd9be9cd0  user/servers/ClockServer.h
    17927bdd3663deee85ddb9224d147384  user/servers/ClockServer.c
    1c77c18014d4834bf7d8cc1e110f3639  user/servers/ClockNotifier.c
    d57a0c5e6aa62c3c2c2f0c5ba2360555  user/Locomotive.c
    a01d7f95458f44436a20e7ffc626b890  user/IdleTask.c
    cbb0a6db7a3fa6c7758e2907aa62e831  user/services/services.c
    4e54be7abd939a75df1d64f2d840ef5e  user/services/terminal.c
    408320cee05a6ef9709468845779a400  user/services/clockService.h
    4e976b1c77a43d5dea7640299506e9e5  user/services/trainCommand.c
    8d1f82b30788352ba85026c0ee136cd6  user/services/nameService.h
    5238b767be62191fd89db795ebc9da79  user/services/trainCommand.h
    9abbea9cf81b40755d6176a08bb9cc73  user/services/services.h
    bdf4cdc781bc41b9684787c8fa4b7ad3  user/services/terminal.h
    b1d37cb42c79cdaa0b5728477f5dad9b  user/services/nameService.c
    66133f586027d068e5254bfefefe62ed  user/services/clockService.c
    f6225db7856a4204228c34dec5a7ea06  user/DigitalClock.h
    f720cb5f7c54893a176e1ffb53a53845  user/SensorDisplay.h
    60df8cdb2606a35d5d4927de9b181c36  user/PerformanceTask.h
    e7ff85ba8f14ce99e0229fff7dd87d52  user/IdleTask.h
    c5351116a601c3683097c17d843d54ee  user/InitialTask.c
    bccd0e2dae1e9fa0e3547810d26222b7  user/messageTypes.h
\end{verbatim}

\end{document}
